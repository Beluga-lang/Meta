\usepackage[total={155mm,200mm},
            top=40mm,
            left=30mm]{geometry}
\usepackage[T1]{fontenc}
\usepackage{microtype}
\usepackage{fix-cm}
\usepackage{amsmath}
\usepackage{amsthm}
\usepackage{amssymb}
\usepackage{amsfonts}
\usepackage{listings}
\usepackage{srcltx}
\usepackage{charter}
\usepackage{euler}
\usepackage{enumerate}
\usepackage{latexsym}
\usepackage{comment}
\usepackage{color}
\usepackage{proof}
\usepackage{url}
\usepackage{xspace}
\usepackage{natbib}
\usepackage{cdsty}


% ---------------------------------------------------------------------------
% ---------------------------- Names of languages ---------------------------
% ---------------------------------------------------------------------------

\newcommand{\beluga}{\textsc{beluga}\xspace}
\newcommand{\twelf}{\textsc{twelf}\xspace}
\newcommand{\minml}{\textsc{minml}\xspace}


% ---------------------------------------------------------------------------
% ------------------------ Theorems and environments ------------------------
% ---------------------------------------------------------------------------

\newtheorem{@problem}{Exercise}[section]
\newenvironment{problem}{\begin{@problem}\rm}{\end{@problem}}
\newtheorem{@sol}{Solution}[section]
\newenvironment{sol}{\begin{@sol}\rm}{\end{@sol}}
\newtheorem{@axiom}{Axiom}
\newenvironment{axiom}{\begin{@axiom}\rm}{\end{@axiom}}

\newtheorem{definition}{Definition}[section]
\newtheorem{theorem}{Theorem}[section]
\newtheorem{conjecture}[theorem]{Conjecture}
\newtheorem{corollary}[theorem]{Corollary}
\newtheorem{proposition}[theorem]{Proposition}
\newtheorem{lemma}[theorem]{Lemma}


% ---------------------------------------------------------------------------
% ------------------------------ Contextual ML ------------------------------
% ---------------------------------------------------------------------------

\lstdefinelanguage{ContextualML}
{
  morekeywords={and, block, case, of, mlam, fn, impossible, let, in, schema,
    some, rec, type, ctype, prop, stratified, inductive, coinductive, LF, if, then,
    else, total},
  keepspaces=true,
  sensitive,
  morecomment=[n]{(*}{*)},
  morestring=[b]"
}[keywords,comments,strings]

\lstloadlanguages{ContextualML}
\lstset{language=ContextualML}

\definecolor{dRed}{rgb}{0.65, 0.0, 0.0}
\definecolor{dGreen}{rgb}{0.133, 0.56, 0.0}

%
% Set up listings "literate" keyword stuff (for \lstset below)
%
\newdimen\zzlistingsize
\newdimen\zzlistingsizedefault
\zzlistingsizedefault=9.5pt
\zzlistingsize=\zzlistingsizedefault
\global\def\InsideComment{0}
\newcommand{\Lstbasicstyle}{\fontsize{\zzlistingsize}{1.05\zzlistingsize}\ttfamily}
\newcommand{\keywordFmt}{\fontsize{0.9\zzlistingsize}{1.0\zzlistingsize}\bf}
\newcommand{\smartkeywordFmt}{\if0\InsideComment\keywordFmt\fi}
\newcommand{\commentFmt}{\def\InsideComment{1}\fontsize{0.95\zzlistingsize}{1.0\zzlistingsize}\rmfamily\slshape}

\newcommand{\LST}{\setlistingsize{\zzlistingsizedefault}}
\newlength{\zzlstwidth}

\newcommand{\setlistingsize}[1]{
  \zzlistingsize=#1%
  \settowidth{\zzlstwidth}{{\Lstbasicstyle~}}}
\setlistingsize{\zzlistingsizedefault}

% The order of the "literate" definitions is significant:
%   later definitions shadow earlier ones.  The \\Pi definition must come
%   *after* the \\ definition, or the first part of \\Pi --- that is, \\ --- will
%   be matched, and instead of $\Pi$ you'll get $\lambda Pi$.
%
\lstset{literate={->}{{$\rightarrow~$}}2 %
                 {=>}{{$\Rightarrow~$}}2 %
                 {|-}{{$\vdash\,$}}2 %
                 {id}{{{\smartkeywordFmt id}}}1 % 3 $~$
                 {\\}{{$\lambda$}}1 %
                 {\\Pi}{{$\Pi$}}1 %
                 {\\gamma}{{$\gamma$}}1 %
                 {\\psi}{{$\psi$}}1 %
                 {FN}{{$\Lambda$}}1 %
                 % {<<}{\color{myotherblue}}1 %
                 {<<}{\color{dGreen}}1 %
                 {<<r}{\color{dRed}}1 %
                 {<*}{\color{dGreen}}1 %
                 {<dim}{\color{dimgrey}}1 %
                 {>>}{\color{black}}1 %
                 {?}{\bf{?}}1
                 ,
                 columns=[l]fullflexible,
                 basewidth=\zzlstwidth,
                 basicstyle=\Lstbasicstyle,
                 keywordstyle=\keywordFmt,
                 identifierstyle=\relax,
                 % stringstyle=\relax,
                 commentstyle=\commentFmt,
                 breaklines=true,
                 breakatwhitespace=true,   % doesn't do anything (?!)
                 mathescape=true,   % interprets $...$ in listing as math mode
                 % tabsize=8,
                 texcl=false}


% ---------------------------------------------------------------------------
% ----------------------- Standard math/CS notations ------------------------
% ---------------------------------------------------------------------------

\newcommand{\subtype}{\leq}

\newcommand{\union}{\mathrel{\cup}}
\newcommand{\sect}{\mathrel{\cap}}
\newcommand{\unit}{\texttt{()}}
\newcommand{\bang}{\texttt{!}}
\renewcommand{\gets}{\mathop{\texttt{:=}}}

\newcommand{\down}{\mathrel{\,\Downarrow\,}}
\newcommand{\step}{\mathrel{\,\Rightarrow\,}}

\newcommand{\syn}{\mathrel{\,\Rightarrow\,}}
\newcommand{\chk}{\mathrel{\,\Leftarrow\,}}
\newcommand{\arr}{\mathrel{\texttt{->}}}
\newcommand{\entails}{\vdash}
\newcommand{\such}{~|~}

\newcommand{\unif}{\doteq}
\newcommand{\totp}{\Rightarrow}
\newcommand{\emp}{\emptyset}

\newcommand{\sectty}{\mathrel{\text{\&}}}

\newcommand{\D}{{\mathcal{D}}}
\newcommand{\C}{{\mathcal{C}}}
\newcommand{\E}{{\mathcal{E}}}
\newcommand{\F}{{\mathcal{F}}}
\newcommand{\V}{{\mathcal{V}}}
\newcommand{\W}{{\mathcal{W}}}
\newcommand{\St}{{\mathcal{S}}}

\newcommand{\edot}{\bullet}

% \newcommand{\arrow}{\rightarrow}

% \newcommand{\chk}{\Leftarrow}
% \newcommand{\syn}{\Rightarrow}
% \newcommand{\bnfas}{\mathrel{::=}}
% \newcommand{\bnfalt}{\mathrel{\mid}}

%\newcommand{\shift}{\textsf{shift}\;}
\newcommand{\shiftn}[2]{\uparrow^{#1}\!#2}
\newcommand{\shift}[1]{\shiftn{}{#1}}
\newcommand{\app}{\;}

\renewcommand{\bnfas}{\;\mathrel{::=}\;}
\renewcommand{\bnfalt}{\, \mid \,}
\newcommand{\lamdb}{\lam\;} % Lambda for de Bruijn

% ---------------------------------------------------------------------------
% ------------------------ Judgments, properties, types ---------------------
% ---------------------------------------------------------------------------

\newcommand{\Int}{\textsf{int}}
\newcommand{\Float}{\textsf{float}}
\newcommand{\Bool}{\textsf{bool}}
\newcommand{\Real}{\textsf{real}}
\newcommand{\String}{\textsf{string}}
\newcommand{\Char}{\textsf{char}}
\newcommand{\Nat}{\textsf{nat}}
\newcommand{\Unit}{\textsf{unit}}
\newcommand{\Ref}{~\textsf{ref}}
\newcommand{\Array}{~\textsf{array}}

\newcommand{\Value}{~\mathsf{value}}
\newcommand{\NumValue}{~\mathsf{num~value}}
\newcommand{\Halts}{~\mathsf{halts}}
\newcommand{\Steps}{\mathrel{\,\longrightarrow\,}}
\newcommand{\MSteps}{\longrightarrow^*}
\newcommand{\Translates}{\leadsto}
\newcommand{\ShiftBy}[3]{\mathsf{shift}\;#1\;#2\;#3}

\newcommand{\FV}{\mathsf{FV}}
\newcommand{\FMV}{\mathsf{FMV}}


% ---------------------------------------------------------------------------
% ------------------------ Terms for object languages -----------------------
% ---------------------------------------------------------------------------

\newcommand{\tmtrue}{\textsf{true}}
\newcommand{\tmfalse}{\textsf{false}}
\newcommand{\tmif}[3]{\textsf{if\;} #1 \textsf{\;then\;} #2 \textsf{\;else\;} #3}
\newcommand{\tmfun}[3]{\textsf{fun } #1 (#2) = #3}
\newcommand{\tmfn}[2]{\textsf{fn } #1\;\texttt{=>}\;#2}
\newcommand{\tmrectyp}[3]{\textsf{rec } {#1}\,:\,{#2}\;\texttt{=>}\;#3}
\newcommand{\tmrec}[2]{\textsf{rec } {#1}\texttt{=>}\;#2}
\newcommand{\tmlet}[3]{\textsf{let } #1 = #2 \textsf{\;in\;} #3\; \textsf{end}}

\newcommand{\tmapp}[2]{\textsf{app}\;#1\;#2}
\newcommand{\tmlam}[3]{\textsf{lam}\;#1{:}#2.#3}
\newcommand{\tmhastype}[2]{\textsf{hastype}\;#1\;#2}

\newcommand{\tmarr}[2]{\textsf{arr}\;#1\;#2}

\newcommand{\tmfst}[1]{\textsf{fst}\;{#1}\xspace}
\newcommand{\tmsnd}[1]{\textsf{snd}\;{#1}\xspace}

% Numerical expressions
\newcommand{\tmzero}{\textsf{z}}
\newcommand{\tmsucc}[1]{\textsf{succ}~#1}
\newcommand{\tmpred}[1]{\textsf{pred}~#1}
\newcommand{\tmiszero}[1]{\textsf{iszero}~#1}

\newcommand{\bref}{\textsf{reflexive}}
\newcommand{\bmred}{\textsf{one{-}step}}
\newcommand{\btrans}{\textsf{transitive}}

\newcommand{\bred}{\beta{-}\textsf{red}}
\newcommand{\blam}{\textsf{lam{-}red}}
\newcommand{\appMred}{\textsf{app{-}red{-}1}}
\newcommand{\appNred}{\textsf{app{-}red{-}2}}

\newcommand{\type}{\textsf{type}}


% ---------------------------------------------------------------------------- %
% -------------------------- Inference rules' names -------------------------- %
% ---------------------------------------------------------------------------- %

\newcommand{\ruleName}[1]{\textsc{\footnotesize #1}\xspace}

\newcommand {\TIf}           {\ruleName{T-If}}
\newcommand {\TPred}         {\ruleName{T-Pred}}
\newcommand {\TZero}         {\ruleName{T-Zero}}
\newcommand {\TPlus}         {\ruleName{T-Plus}}
\newcommand {\TMult}         {\ruleName{T-Mult}}
\newcommand {\TEq}           {\ruleName{T-Eq}}
\newcommand {\TApp}          {\ruleName{T-App}}
\newcommand {\TLam}          {\ruleName{T-Lam}}
\newcommand {\TSub}          {\ruleName{T-Sub}}
\newcommand {\TFn}           {\ruleName{T-Fn}}
\newcommand {\TFun}          {\ruleName{T-Fun}}
\newcommand {\TPair}         {\ruleName{T-Pair}}
\newcommand {\TFst}          {\ruleName{T-Fst}}
\newcommand {\TSnd}          {\ruleName{T-Snd}}
\newcommand {\TVar}          {\ruleName{T-Var}}
\newcommand {\TNum}          {\ruleName{T-Num}}
\newcommand {\TTrue}         {\ruleName{T-True}}
\newcommand {\TFalse}        {\ruleName{T-False}}

\newcommand {\TBinaryPrimop} {\ruleName{T-Binary-Primop}}
\newcommand {\TUnaryPrimop}  {\ruleName{T-Unary-Primop}}
\newcommand {\TTuple}        {\ruleName{T-Tuple}}
\newcommand {\TTupleSyn}     {\ruleName{T-Tuple-Syn}}
\newcommand {\TRec}          {\ruleName{T-Rec}}
\newcommand {\TAnno}         {\ruleName{T-Anno}}

\newcommand {\TrLam}         {\ruleName{Tr-Lam}}
\newcommand {\TrApp}         {\ruleName{Tr-App}}
\newcommand {\TrTop}         {\ruleName{Tr-Top}}
\newcommand {\TrNext}        {\ruleName{Tr-Next}}

\newcommand {\TLet}          {\ruleName{T-Let}}
\newcommand {\TLetSyn}       {\ruleName{T-Let-Syn}}
\newcommand {\TDecs}         {\ruleName{T-Decs}}
\newcommand {\TByName}       {\ruleName{T-By-Name}}
\newcommand {\TByVal}        {\ruleName{T-By-Val}}
\newcommand {\TByValTuple}   {\ruleName{T-By-Val-Tuple}}

\newcommand {\EIfTrue}       {\ruleName{E-IfTrue}}
\newcommand {\EIfT}          {\EIfTrue}                          % to be deleted
\newcommand {\EIfFalse}      {\ruleName{E-IfFalse}}
\newcommand {\EIfF}          {\EIfFalse}                         % to be deleted
\newcommand {\EIf}           {\ruleName{E-If}}
\newcommand {\ESucc}         {\ruleName{E-Succ}}
\newcommand {\ESuc}          {\ESucc}                            % to be deleted
\newcommand {\EPred}         {\ruleName{E-Pred}}
\newcommand {\EPredZero}     {\ruleName{E-PredZero}}
\newcommand {\EPredZ}        {\EPredZero}                        % to be deleted
\newcommand {\EPredSucc}     {\ruleName{E-PredSucc}}
\newcommand {\EIszero}       {\ruleName{E-Iszero}}
\newcommand {\EIsZero}       {\EIszero}                          % to be deleted
\newcommand {\EIszeroZero}   {\ruleName{E-IszeroZero}}
\newcommand {\EIsZeroZ}      {\EIszeroZero}                      % to be deleted
\newcommand {\EIszeroSucc}   {\ruleName{E-IszeroSucc}}
\newcommand {\EIsZeroSucc}   {\EIszeroSucc}                      % to be deleted

\newcommand {\EAppFnStep}    {\ruleName{E-App1}}
\newcommand {\EAppArgStep}   {\ruleName{E-App2}}
\newcommand {\EAppBeta}      {\ruleName{E-App-Abs}}

\newcommand {\MRef}          {\ruleName{M-Ref}}
\newcommand {\MTr}           {\ruleName{M-Tr}}
\newcommand {\MStep}         {\ruleName{M-Step}}

\newcommand {\NVZero}        {\ruleName{Nv-Zero}}
\newcommand {\NVSucc}        {\ruleName{Nv-Succ}}

\newcommand {\VNumValue}     {\ruleName{V-NumValue}}
\newcommand {\VZero}         {\ruleName{V-Zero}}
\newcommand {\VTrue}         {\ruleName{V-True}}
\newcommand {\VFalse}        {\ruleName{V-False}}
\newcommand {\VSuc}          {\ruleName{V-Succ}}                 % to be deleted

\newcommand {\BAnno}         {\ruleName{B-Anno}}
\newcommand {\BAnnoFn}       {\ruleName{B-Anno-Fn}}
\newcommand {\BAnnoNonFn}    {\ruleName{B-Anno-Non-Fn}}
\newcommand {\BIFT}          {\ruleName{B-IfTrue}}
\newcommand {\BIFF}          {\ruleName{B-IfFalse}}
\newcommand {\BOp}           {\ruleName{B-Op}}
\newcommand {\BPlus}         {\ruleName{B-Plus}}
\newcommand {\BEq}           {\ruleName{B-Eq}}
\newcommand {\BLet}          {\ruleName{B-Let}}
\newcommand {\BNum}          {\ruleName{B-Num}}
\newcommand {\BVar}          {\ruleName{B-Var}}
\newcommand {\BIF}           {\ruleName{B-If}}
\newcommand {\BTrue}         {\ruleName{B-True}}
\newcommand {\BFalse}        {\ruleName{B-False}}
\newcommand {\BFun}          {\ruleName{B-Fun}}
\newcommand {\BRec}          {\ruleName{B-Rec}}

\newcommand {\BLetn}         {\ruleName{B-Letn}}
\newcommand {\BLetp}         {\ruleName{B-LetPair}}
\newcommand {\BPair}         {\ruleName{B-Pair}}
\newcommand {\BFst}          {\ruleName{B-Fst}}
\newcommand {\BSnd}          {\ruleName{B-Snd}}
\newcommand {\BFn}           {\ruleName{B-Fn}}
\newcommand {\BApp}          {\ruleName{B-App}}

\newcommand {\Rsectintro}    {\ruleName{$\sectty$intro}}
\newcommand {\Rsectelim} [1] {\ruleName{$\sectty$elim{#1}}}


% ---------------------------------------------------------------------------
% ----------------------------------- ??? -----------------------------------
% ---------------------------------------------------------------------------

\newcommand{\hasty}[2]{#1~\mathsf{hastype}~#2}
\newcommand{\jtype}{\mathit{type}}


\newcommand{\mdots}{\,.\hspace{-0.025cm}.\hspace{-0.025cm}.\,}


\newcommand{\eq}[2]{\;\mathtt{eq}\;#1\;#2}

\newcommand{\lb}{{[\![}}
\newcommand{\rb}{{]\!]}}

\newcommand{\nl}{\overline{n}}

\renewcommand{\Tilde}{\textsf{\char"7E}}

\bibpunct{[}{]}{;}{a}{}{,}
\def\newblock{\hskip .11em plus .33em minus .07em}

\newcommand{\arrayenv}[1]{\renewcommand{\arraystretch}{1} \begin{array}[t]{@{}c@{}}#1\end{array}}
\newcommand{\arrayenvc}[1]{\renewcommand{\arraystretch}{1} \begin{array}[c]{@{}c@{}}#1\end{array}}
\newcommand{\arrayenvr}[1]{\renewcommand{\arraystretch}{1} \begin{array}[t]{@{}r@{}}#1\end{array}}
\newcommand{\arrayenvbr}[1]{\renewcommand{\arraystretch}{1} \begin{array}[b]{@{}r@{}}#1\end{array}}
\newcommand{\arrayenvl}[1]{\renewcommand{\arraystretch}{1} \begin{array}[t]{@{}l@{}}#1\end{array}}
\newcommand{\arrayenvb}[1]{\renewcommand{\arraystretch}{1}  \begin{array}[b]{@{}c@{}}#1\end{array}}
\newcommand{\arrayenvbl}[1]{\renewcommand{\arraystretch}{1}  \begin{array}[b]{@{}l@{}}#1\end{array}}

\renewcommand{\t}[1]{\lstinline!#1!}
\newcommand{\bel}[1]{\lstinline!#1!}

% ---------------------------------------------------------------------------
% ------------------- MAGIC (a.k.a. Work in Progress) -----------------------
% ---------------------------------------------------------------------------

\newcommand{\proofderiv}[2]{\mathbin{#1 \; :: \; #2 }}
\newcommand{\proofderivc}[3]{\proofderiv{#1}{#2 \vdash #3}}
\newenvironment{case}[1]{\paragraph{Case}{#1}\\[1em]}{}
\newenvironment{basecase}[1]{\paragraph{Base case}{#1}\\[1em]}{}
\newenvironment{stepcase}[1]{\paragraph{Step case}{#1}\\[1em]}{}
\newenvironment{subcase}[1]{\textbf{Subcase}{\quad #1}\\[0.5em]}{}

\newcommand{\infera}[3]{\ianc{#3}{#2}{#1}}
\newcommand{\inferaa}[4]{\ibnc{#3}{#4}{#2}{#1}}
\newcommand{\inferaaa}[5]{\ibnc{#3}{#4}{#5}{#2}{#1}}

% ---------------------------------------------------------------------------
% --------------------------------- Other -----------------------------------
% ---------------------------------------------------------------------------

\newcommand{\emphFact}[1]{{\color{dGreen}{#1}}}
