\chapter*{Preface}
This book provides an introduction to mechanizing the meta-theory of programming languages. The exposition is intended for a broad range of readers, from advanced undergraduates to PhD students and researchers.  The book is written as a companion to B. Pierce's book ``Types and Programming Languages (TAPL)''that provides an introduction to how to mechanize the meta-theory of types and programming languages. While it is meant to be read  at the same time as TAPL, we provide enough context and background that it should also be easily accessible to a reader who has read TAPL or has already basic knowledge of types and programming languages. 

Here is a roadmap:

% ADD FIGURE

The material covers about one semester's worth of material and has been used at McGill University for teaching the course ``Language-based security'', a course open to advanced undergraduates and beginning graduate students. 


We have chosen to mechanize in Beluga, a dependently typed programming and proof environment as it directly supports key and common concepts that frequently arise when describing formal systems and derivations within them; in particular it provides infrastructure for modelling variable binders and their scope, it supports first-class contexts to abstract, manage, and manipulate a set of assumptions, it support modelling derivations that depend on assumptions, and has built-in first-class (simultaneous) substitutions.  


\paragraph{Note to instructors}
If you intend to use this material for your own course, we would love to hear from you. We welcome all comments, questions, and suggestions.

\paragraph{Acknowledgements}